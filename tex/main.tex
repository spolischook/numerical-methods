\documentclass[12pt,a4paper]{article}
\usepackage{minted,xcolor}
\usemintedstyle{manni}
\definecolor{bg}{HTML}{ffffff}
\usepackage{mathtext}
\usepackage{listings}
\usepackage{color}
\usepackage{float}
\usepackage{graphicx}
\usepackage[T1,T2A]{fontenc}
\usepackage[utf8x]{inputenc}
\usepackage[english,ukrainian]{babel}
\usepackage[a4paper, total={7in, 9.5in}]{geometry}

\usepackage{listings}
\usepackage{lipsum}

\title{Числові методи ЕОМ}
\selectlanguage{ukrainian}
\author{Сергій Поліщук}
\date{1984/02/14}
\selectlanguage{english}
\begin{document}

\maketitle
\tableofcontents

\clearpage
\section{Розв'язування нелінійних рівнянь з однією змінною}
\subsection{Розбиття на відрізки} \label{subsection}
\subsection{Метод ітерацій} \label{subsection}
\subsection{Метод дихотомії} \label{subsection}
\subsection{Метод Нютона} \label{subsection}

\clearpage
\section{Резонансні частоти для заданих коефіцієнтів}

\clearpage
\section{Методи знаходження площі інтеграла}
\subsection{Аналітичний метод}
Аналітичний метод зводиться до обчислення визначенного інтегралу за формулою:
\begin{equation} \label{eq:1}
\int_{a}^{b}\frac{1}{1+x}dx = ln\left ( \frac{1+b}{1+a} \right )
\end{equation}
Що для відрізку ві 0 до 3 дорінює 1.38629436111989
\subsection{Метод прямокутників} \label{integralRectangle}

Вісь X розбивається на скінченну кількість рівних відрізків. 
По кожному X знаходиться значення функції (Y) і обраховується площа такого прямокутника
\begin{figure}[ht]
    \centering
    \def\svgwidth{\columnwidth}
    \caption{Графічна інтерпретація методу прямокутників}\label{integralRectangleFigure}
    \input{integralS_rectangle.pdf_tex}
\end{figure}

На графіку ~\ref{integralRectangleFigure} кожен прямокутник одним кутом виходить за графік, саме в цьому проявляється похибка розв'язку. Чим меньші відрізки по вісі X тим менша похибка.

\begin{listing}[ht]
\inputminted[
	firstline=9, 
	lastline=28, 
	linenos, 
	bgcolor=bg,
    baselinestretch=1.2,
    framesep=2mm]{R}{../includes/integralS.R}
\caption{Метод прямокутників}\label{integralRectangleListing}
\label{listing:1}
\end{listing}

В модулі integralS.R реалізація методу прямокутників інкапсульована у функції integralS.rectangle. Функція приймає 4 обов'язкових аргумента, та одни не обов'язковий plot, що відповідає за візуалізацію метода

\clearpage
\subsection{Метод трапецій}

\begin{figure}[ht]
    \centering
    \def\svgwidth{\columnwidth}
    \caption{Графічна інтерпретація методу трапецій}\label{integralTrapezeFigure}
    \input{integralS_trapeze.pdf_tex}
\end{figure}

\begin{listing}[ht]
\inputminted[
	firstline=27, 
	lastline=43, 
	linenos, 
	bgcolor=bg,
    baselinestretch=1.2,
    framesep=2mm]{R}{../includes/integralS.R}
\caption{Метод трапецій}
\label{listing:1}
\end{listing}

\clearpage
\subsection{Метод Симпсона} \label{subsection}

\begin{figure}[ht]
    \centering
    \def\svgwidth{\columnwidth}
    \caption{Графічна інтерпретація методу Сімпсона}\label{integralSimpsonFigure}
    \input{integralS_simpson.pdf_tex}
\end{figure}

\begin{listing}[ht]
\inputminted[
	firstline=57, 
	lastline=74, 
	linenos, 
	bgcolor=bg,
    baselinestretch=1.2,
    framesep=2mm]{R}{../includes/integralS.R}
\caption{Метод Сімпсона}
\label{listing:1}
\end{listing}


\subsection{Метод Монте-Карло} \label{subsection}
\begin{listing}[ht]
\inputminted[
	firstline=30, 
	lastline=52, 
	linenos, 
	bgcolor=bg,
    baselinestretch=1.2,
    framesep=2mm]{R}{../includes/integralS.R}
\caption{Метод Монте-Карло}
\label{listing:1}
\end{listing}

\clearpage
\section{Інтерполяційний многочлен Лагранжа}

\clearpage
\section{Висновки}

  






\clearpage
\section{Лістинг коду} \label{subsection}
\renewcommand\listoflistingscaption{}
\listoflistings

\end{document}
